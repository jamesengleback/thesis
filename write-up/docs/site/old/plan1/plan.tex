\documentclass{article}
\usepackage{parskip}
\usepackage{fontspec}
\usepackage{graphicx}
\setmainfont{Noto Sans}

\author{James Engleback}
\begin{document}
\title{\textbf{Plan and schedule for 2021}}
\maketitle

\tableofcontents
\section{Overview}
\par

This project has two peices of work both with the aim to engineer a P450 BM3 mutant with some sort of metabolic activity towards the herbicide mesotrione, ideally hydroxylation at carbon 5. % intro
One approach uses classical structure prediction and docking combined with a genetic algorithm to generate mutantions with predicted activity. % outline enz work
Prediction of BM3-mutant binding will be used as a fitness metric in virtual directed evolution experiments. The resulting mutants will be made and tested for predicted activity in the lab. % engineering
\par
The other uses a deep learning model trained on screening data of BM3 mutant binding with herbicide-like compounds, with the aim of being able to predict the activity of any BM3 mutant with any herbicide.  % outline rio work
This work involves high-throughput screening of BM3 mutants against herbicide-like compounds. % screening
Mutants resulting from virtual directed evolution experiments using the deep learning model for fitness estimation will be made in the lab and tested for predicted activity. % engineering
\par
This document outlines each approach and a schedule for work to complete and write up each into a thesis by the September deadline. % schedule and this document
\par
\subsection{Target}
Mesotrione is an important herbicide whose metabolism in plants is initiated by hydroxylation of carbon 5 by a P450. Given the promiscuity of glutathione-S transferases that help sequester a hydroxylated xenobiotic, alternative hydroxylation sites may be sufficient to initiate herbicide metabolism in plants. % mesotrione
A BM3 mutant that hydroxylates mesotrione at any position may be suitable for engineering mesotrione resistance into a crop. % hydroxylate at any position
\par
\subsection{Prior work}
The promiscuous BM3 A82F/F87V mutant shows no binding or catalytic activity towards mesotrione despite activity towards several less important herbicide classes. It may not be possible to create a BM3 mutant with desired mesotrione activity, so this work is motivated towards creating enzyme design systems that generalize to other herbicides, such as herbicides in development. % other herbicides
\par
\subsection{Approaches}
The two approaches developed here are based on traditional structure-based methods and one based on machine learning. % approaches
Both approaches will produce a pool of BM3 variants with predicted binding activity towards mesotrione, for whom $K_d$, $K_{cat}$, $K_M$ and product formation will be measured. % similarities
In this document, this project is referred to as \textit{enz}. % abbreviations

\subsubsection{Structure-based virtual screening - overview}
Structure based design relies on template-based structure prediction to generate mutant structures and molecular docking to evaluate the fitness of mutants for a genetic algorithm, which steers a virtual directed evolution process. % enz
\par
\subsubsection{Artificial Intelligence-based design - overview}
An artificial intelligence-based approach uses a deep neural network to predict the likelihood of binding between input amino acid sequences and chemical smiles and design lab experiments. % outline
The network is trained on data of $K_d$ between 5-10 BM3 mutants and around 100 herbicide-like compounds, and will be used to make $K_d$ predictions for pairs of amino acid sequences and compound. %model 
The screening data will be generated in the lab using a high-throughput UV-Vis assay for measuring $K_d$. % screening
\par
The trained model can estimate $K_d$ rapidly and at scale, so will be used for large-scale virtual directed evolution experiments selecting for mesotrione affinity to generate pools of new BM3 mutants to be made in the lab and tested for predicted activity. % use model
\par
Both approaches generate pools of BM3 mutant sequences with predicted binding activity for mesotrione. The lab validation process is:% lab evaluation
\begin{enumerate}
	\item \textbf{Preparation of expression plasmids} will be done using site-directed mutagenesis. Several BM3 mutant plasmids are in stock and can serve as starting points. The search depth of the models will be limited to reduce the number of mutagenesis steps. % plasmids
	\item \textbf{Expression} of BM3 mutants in \textit{E. coli} is fairly robust. Two mutants per shaker should yeild sufficient protein for testing. The full length protien will be expressed.
	\item \textbf{Purification} by a single Nickel-affinity chromatography step is sufficient to reliably yeild BM3 mutants at a purity suitable for measuring $K_{cat}$, $K_d$, $K_m$ and product formation.
	\item \textbf{Measurement} 
		\begin{itemize}
			\item \textbf{$K_d$} will be measured by titration of compound into BM3 with UV-Vis measurements. This experiment typically lasts 20 minutes.
			\item \textbf{$K_m$ and $K_{cat}$} will be measured by monitoring NADPH consumption at 340 nm.
			\item \textbf{Product formation} will be detected my LCMS of a turnover reaction. Indication of any mesotrione metabolite is the target. Further elucidation is future work.
		\end{itemize}
\end{enumerate}
\par
In this document, this project is referred to as \textit{rio}. % abbreviations
\par
Both approaches are being built into \textit{Python} packages to ensure portability to other enzyme engineering problems. % packages

\section{Structure-based design}

\subsection{Aim and overview}
Template x-ray crystal structures for BM3 are plentiful. The BM3 active site is mostly helical and is fairly temperature stable, so low-resolution template-based structure prediction may be suffuciently accurate for use in fitness evaluation in a virtual directed evolution experiment. % justification
\par
For my own convenience and reusability, structure-prediction and molecular docking functions used in this work have been packaged into the \textit{Python} module \textit{enz}. Packaging has enabled large-scale virtual directed evolution experiments to be set up easily. % package justification


\subsubsection{\textit{enz} - a \textit{Python} package for enzyme design}
\textit{enz} is a \textit{Python} Application Program Interface (API) I've made for this work that automates template-based structure prediction and docking. It can be used for virtual directed evolution by combination with a genetic algorithm. Docking calls \textit{autodock vina} and refolding calls \textit{pyrosetta} % work done
\textit{enz} is simple to use, is potentially useful to other protein engineers and has potential for further development. % use
\par
The package works and is ready to deploy in a virtual directed evolution experiment using a custom docking score as a fitness metric and a simple geneteic algorithm to breed and mutate BM3 variants over around ten rounds. % readiness and genetic algorithm
Improvements to \textit{enz} can be made to imporove accuracy: % enz to do features
\begin{itemize}
	\item \textbf{flexible docking} - currently side chains are modelled as rigid which limits docking accuracy. \textit{vina} is capable of flexible docking, so an attempt to implement it should be made. 
	\item \textbf{loop remodelling} - currently conformation of loops and flexible regions are not recalculated, this would use the cyclic coordinate descent (CCD) implementation in \textit{pyrosetta} for all unstructured loops within a cutoff radius of a mutation. CCD is a relatively fast algorithm and there in BM3 some important active site mutations are in loops.
\end{itemize}
Ideally support would be added for these non-features before running the genetic algorithm with \textit{enz}. I am allocating a day to attempt implementing each of the proposed improvements. % to do
\par

\subsubsection{\textit{enz} validation}
Accuracy of \textit{enz} will be assessed by replicating ligand-bound BM3 structures from the PDB in \textit{enz}.  % validation overview
In the experiment, a set of ligand-bound BM3 crystal structres are selected to be replicated using \textit{enz} using structure prediction from a suitable template structure and docking to predict faseible ligand binding poses. % experiment
% todo which mutants?
RMSD ($C_{\alpha}$) between the predicted and experimentally determined BM3 structures will to indicate \textit{enz}'s structure prediction accuracy and inform features top be added to \textit{enz} in future work.  % structure
RMSD between ligand positions in both the PDB structures and the \textit{enz} predictions will indicate accuracy. % ligands
\par


\subsubsection{Designing new mutants with \textit{enz}}
\textit{enz} works well enough to use and is ready to deploy with a simple genetic algorithm for mutant generation. Since the active site of BM3 has only 4 short (3-6 residues)  flexible regions in the active site, the impact of loop remodelling may be small and despite not docking ligands with flexible protein side chains, the results may be sufficient to generate a pool of mutants to be tested in the lab. However implementation of flexible docking and loop remodelling is very desirable. % running as is
\par
The heuristic currently employed to estimate the desirability of each set of docking results is % heuristic
\begin{equation}
score = \frac{1}{n}\Sigma _{n} \Delta G_{n} \times d_{n}
\end{equation}
where $ \Delta G $ is free energy of the interaction calculated by \textit{autodock vina} (\textit{kcal / mol}) and $ d $ is the distance between the heme iron and the C5 of mesotrione for $ n $ in binding poses. Where C5 is the target carbon for hydroxylation. It may be sufficient for use in the genetic algorithm but can be flexibly changed if needed. % heuristic
Improvements to this heuristic for future work would include RMSD to a favourably positioned reference pose and implementing a machine learning-based score function for more accurate binding energy predictions. % future todo reference rf and nn score and others
\par
The runtime of the genetic algorithm should not exceed 2 days, even on a small virtual private server. A pool of mutants yielded by genetic algorithm-based virtual directed evolution will be made in the lab for testing activity towards mesotrione. % validation process
The pool size of the mutants to make and test will be constrained by time and resource. The pool will be made in the lab and tested for $K_d$ and $K_{cat}$ with mesotrione and an incidation of product formation. % validation process 

\subsubsection{Validation of mutants}
Mutants generated by the genetic algorith \textit{enz} combination will be made and tested in the lab with three techniques:% validation intro
\begin{itemize}
\item \textbf{Mesotrione titration} to get a $K_d$ if any.
\item \textbf{Steady state kinetics} for a $K_{cat}$ and $K_m$ via NAPDH consumption
\item \textbf{LCMS} product detection. A +16 \textit{m/z} or fragmentation of mesotrione is considered a hit.
\end{itemize}
\par
\subsubsection{Order of events}
The expected order of events for this project is:
% order of enz events
\begin{enumerate}
\item Final changes to \textit{enz} - if possible - timescale: 2 days
\item Validate \textit{enz} against known BM3 crystal structures - 1-2 days
\item Run genetic algorithm, select pool of mutants - runtime < 2 days
\item Prepare the mutants in the lab and test for the proposed metrics - timescale < 3 months
\end{enumerate}
The timings of this operation will be expanded on in the \textbf{schedule} section.

\subsection{Follow-up work}
Short-term (months) future work on structure prediction-based enzyme design using \textit{enz} would see implementation of additional \textit{pyrosetta} structure prediction methods in the \texttt{refold()} function and implementation of high-resolution docking in the \texttt{dock} function using a method provided by \textit{pyrosetta}. % short term todo ref rosetta docking
\par
Further work would see the continued development of \textit{enz} into an open-source python module that does not rely on \textit{pyrosetta}. Folding and docking algorithms would be re-implemented in an open-source tensor math framework like \textit{jax} or \textit{pytorch}, which support parralelization with GPU processing. % future work - enz

A notable advantage of a tensor math backend is compatibility with machine learning-based local structure prediction, which outperforms many traditional methods. The proposed machine learning-based structure predicion method will replicate that of Alphafold 1 and 2, which won CATEGORY in CASPX (2018) and CASPY (2020) respectively. This method relies on a conformation-dependent learned score function of protein structure likelihood and gradient descent to steer confomational changes to maximise the structure likelihood score. This step is notably fast, taking XXXX to XXx with XXXXX. % alphafold gradient descent
% todo alphafold references - back up statements
\par

\section{Machine learning-based design}

\subsubsection{Aim}
The aim of this project is to produce a general solution to BM3 mutant:herbicide metabolism prediction, and use that to generate BM3 mutants with activity towards mesotrione or any other compound. The model aims to use amino acid sequence and chemical smiles alone to make $pK_d$ predictions. % aim - general model
This is a proof of concept method and models in follow-up work would can be trained to estimate additional metrics. % pkd and other outputs 
\par
\cite{MONN:m} is% related work

\subsubsection{Approach}
In this project, the approach to creating the model relies on creating a screening dataset of several BM3 mutants against a planned 100 compounds. % approch - multi screen
\par
To make the model faesible, it must be pre-trained on a large dataset before training on the domain-specific data. This approach is called transfer learning and enables learning from small domain-specific datasets. The Pre-training dataset contains  pairs of P450 amino acid sequences and SMILES codes for reactants for the corresponding Enzyme Comission number. This has been gathered using data mining programs made for this work and currently contains the order of $10^5$ datapoints. The hope is that quantity of data overcomes quality during pre-training. % model and transfer learning todo - ref
% todo split up 
% todo transfer learning figure
\par
The sucess of pre-training will be measured by final model performance on the in-house screening dataset. Search for a suitable pre-training end poind can be automated using meta-learning techniques such as Bayesian optimization. % pre-training end point 

\subsubsection{Model architecture}
The model is based on leading techniques from chemical learning and natural language processing (NLP). The best performing models in chemical learning use graph convolutional neural networks to learn features of molecules, which are employed in this model for molecular input. The best known architecture for sequence learning is the transformer, a model that enjoys superiority in NLP and other fields so is also employed here.  % architecture overview
% todo ref architectures
\par
In more detail, the model constructs a chemical graph from a smiles string, which is convolved by a graph convolutional network and downsampled to a fixed-size embedding vector using set2set. The amino acid sequence string is downsampled using three 1D-convolutional layers and processed to a short, variable-length tensor which is downsampled into a fixed-size vector using the final hidden-state from an LSTM layer. % detail
The learned embedding vectors of the sequence and conatenated into a combined representation vecor. A two ayer perceptron provides a point estimate of binding likelihood. % more detail
\par
At certain points in training, it may be necessary to replace the output head of the model with perceptrons that can generate multiple predictions. Replacing elements of a trained network is common in transfer learing to adapt a model to a new task.

\subsubsection{Pretraining}
\par
The pre-training dataset has been mined from KEGG and currently holds roughly 0.2 million datapoints of enzyme sequence and smiles for reactants and products. I've identified changes to the data miner that will could increase the dataset size close to $10^6$.  % kegg
\par
The sets of substrates for each sample in the KEGG dataset are positive examples only, so for the model to learn the distribution of valid sets of enzyme and compounds it is being trained with a generative adversary. For each datapoint, a generator network generates a synthetic datapoint whilst the model that takes an input of enzyme sequence and molecular graph must discriminate between real and fake. The training loop is formulated as a competitive two player game between the generator and discriminator and results in increasing quality of synthetic data and inference of the underlying data distribution for the discriminator. %% gan loop - todo revisit, ref 

Over training the model will discriminate between real and fake data, whilst the generator will create more realistic samples. The generator can be discarded after training whilst the discriminator can be repurposed for $pK_d$ prediction and other tasks. % generative pre-training
\par
There is scope to monitor training by assessing the quality of the generated sequences and molecular graphs. % monitoring pre training
The extent of pre-training will be deterimend by trails that measure accuracy of the model on screening data when retrained. % end to end
\par
An additional dataset of P450:xenobiotics binding data has been filtered from \textit{BindingDB}. The metrics for each P450:xenobiotic pair are one of $K_d$, $K_{cat}$, $K_i$ and $IC_{50}$. For this task, the output layer of the model will be replaced with a perceptron with an output head for each target metric. This dataset conatins $X$ data points and exposes the model to xenobiotics, which may assist with training on the in-house dataset. % bindingdb dataet
\par
Saved pre-trained models can then be transferred to the screening data task for their accuracy to be assessed. % next stage training

\subsubsection{Pilot screen}
A 384 well plate-based UV-Vis assay for measuring $K_d$ has been developed for this work. Development is ready to write into a \textit{methods} section. It can be set up using a multichannel pipette or an Echo acoustic liquid handling robot. One plate measures $K_d$ between one mutants and 24 compounds at 8 concentrations, including 8 blanks to correct for UV-Vis absorbance of the compounds. One plate can be prepared and read in 20 minutes. Accuracy is limited compared to titration assays, but can identify clear binding signals. % assay
\par
An alternative assay format that measures the UV-Vis response of BM3 to a single high concentration can be readily adopted. This would save X compound and X protein per plate and increase the compound capacity of each plate to 192. % alternative format assay
This alternative may produce data with similar value to $pK_d$ with either a hit or miss end point, or a quantification of BM3 response. It will also increase the number of compounds that can be tested. % justification
\par
%%%%%%%%%% pilot
A pilot screening trial of 100 in-house compounds aginst 5 BM3 mutants has taken place and preliminary results indicate X. % has taken place
The 5 BM3 mutants included:
\begin{itemize}
	\item wild type
	\item a82f/f87v
\end{itemize}
The screening compounds contained a herbicide-like and structurally diverse set of herbicides and  compounds selected from an in-house FDA-approved drug screening library. The library was filtered according to herbicide-likeness rules (ref) and a set of $n$ compounds was selected using a compound diversity picking algorithm (MaxMin). % todo ref herbicide likeness
The screening compounds contained an additional set of $n$ herbicides, including mesotrione.
\par
One outcome of the pilot screen is the evaluation of whether attempting to measure $K_d$ with 8 concentrations of compound gives better quality data than a qualitative hit or miss metric determined with only one concentration. % pilot evaluate - qual or quant
\par
The pilot will also indicate a suitable dataset size for main screening. The training weights learned from the pilot dataset may also be progressed to training on the main screening set. % screening set size
\par
\subsubsection{Screening Library Design}
A suitable size of screening library can be inferred by results from the pilot study. % size
A method for designing a herbicide-like screening library based on size constraints was programmed in this work. Library size is adjusted to budget and based on the \textit{Molport} library - an aggregation of compounds from many suppliers (> 2M). \textit{Molport} ship custom selections of compounds and have an API for generating quotes. % compound library 
\par
I have been working with a sales manager from \textit{Molport} for this, who tells me that the lead time for a library to arrive in the UK is 20 days if the stock compounds and that they can provide a COSHH form that covers all compounds. \textit{Molport} has an API that I have been using to autogenerate quotes and their lead times. The compound selection method from the \textit{Molport} library is the same as that used to select $n$ structurally diverse and herbicide-like compounds.  % molport and ordering
\par
$n$ compounds are ready to be selected from the \textit{Molport} database, where $n$ and $volume$ of compounds an be adjusted to fit buget constraints. $n$ and $volume$ will also depend on the decided assay format, either hit or miss or $pK_d$ measurements. % ordering readiness
% todo size and price estimates
\par
BM3 mutants for screening currently comprise X variants who have crystal structures in the PDB. Purified variants that are ready to screen are:% todo mutant selection
\begin{itemize} % pure mutants
	\item we
\end{itemize}
X mutants have been expressed in \textit{E. coli} and are ready to be purified:
\begin{itemize} % cell pellets
	\item cell pellets
\end{itemize}
\par
Mutants planned for expression, purification and screening are: % planned mutants
\begin{itemize} % planned mutants
	\item planned
\end{itemize}
\par
\subsubsection{Screening}
The mutant BM3 $n$ cell pellets can be purified immediately over the course of two weeks and screened against either in-house herbicide-like FDA-approved compounds or a new libraryselected from \textit{Molport} over the course of another week.  % screening time estimates
\par
Further rounds of mutant screening involve BM3 mutants with known altered substrate specificity can proceed as soon as the mutants are purified. % next round
The proposed mutants for screening are listed in order of priority: % which mutants
\begin{enumerate} % todo - planned mutants for screening
	\item a333s
\end{enumerate}
\par
Structure prediction and docking can be used to indicate altered substrate specificity before synthesis to probe the usefulness of each mutant in a screen. % enz sim of proposed mutants - usefulness of mutations
\par
\subsubsection{Training}
The required dataset size for domain-specific data is not known, but transfer learning gives the best chance of sucess with a small dataset size and the best chance that the model can extrapolate to predictions outside of the screening dataset. The model will either be trained to directly predict $K_d$ between sequence and compound, or continue to predict likelihood of binding interaction in the case that the data is only accurate enough to yield qualative results. % training on screening dataset
\par
I have sufficient access to hardware for this task. Training and parameter selection may last one week. % hardware
\par
After training the model will be evaluated on a validation set left out of the training data. Uncertainty of model predictions can be evaluated and used to determine which areas of input space the model is unsure of and next experiments can be designed accordingly. % evaluation

\subsubsection{Mutant design and testing}
The trained model can deliver predictions very fast and in parralel. To generate mutants a genetic algorith will be used to generate mutants due to its simplicity and parralelizability. The genetic algorithm will generate amino acid strings of BM3 mutants based on constrained mutation sites, using predicted $pK_d$ towards mesotrione as a selection criteria. % mutant generation 
\par
Members of the mutant pool generated by the genetic algorithm will be selected for validation in the lab. A major factor is the number of mutations W.R.T existing plasmid stocks. % mutant pool
\par
As in the structure-based design, the testing process for mutants for a pool of $n$ mutants is: % rio mutant validation
\begin{enumerate}
\item expression and purification using nickel affinity chromatography
\item $K_d$ measurement with mesotrione using titration and UV-Vis spectroscopy
\item $K_{cat}$ measurement with mesotrione by monitoring reaction NADPH consumption by UV-Vis spectroscopy
\item product analysis with LCMS where a hydroxylation at any position is considered a hit
\end{enumerate}

\subsubsection{Follow-up work: Active learning}
\par
Continuation of this project would see the model built into an artificial intelligence system that can design optimal screening experiments using adaptive learning. This could allow full automation of enzyme design using this approach.

\section{Syngenta Placement}
I plan to re-establish communication with Syngenta by sending a report with proof of concept for each project. The \textit{enz} project can be presented in its current state and the \textit{rio} project can be presented after the pilot screen due to complete in Feb. % motivate report
Given this, the deadline for sending this report is Feb 28th. % report schedule
The most useful placement to me is to work remotely with Syngenta computer scientist who have had involvement with this project. Nathan Kidley is a virtual screening specialist who can advise on \textit{enz}-related work and Kostas Papachristos is a machine learining engineer who can advize on \textit{rio} work. Richard Dale and Christian Noble can advise on lab work if required. % virtual placement
\subsection{Follow-up work}
Syngenta have the facilities to transform a P450 into a model crop and spray test it against mesotrione in a glass house. The turn around time for this process is 1-2 months. To publish this work in this follow-up work, it will be important to include a spray test. % future tobacco spray tests

\section{schedule}

\subsection{Feb}
\subsubsection{Structure-based}
A genetic algorithm using \textit{enz} as is ready to be deployed using the heuresic described. % readiness
Improvements to \textit{enz} to improve accuracy by enabling flexible ligand docking and loop remodelling will be added this month. I would benefit from some examples of flexible docking in \textit{autodock vina} using \textit{Openbabel}. % enz improvements
My deadline for running the program is Feb 16th, whatever the state of \textit{enz}.
Primer design for site-directed mutagenesis will be completed by the 17th. % primer design

\subsubsection{Machine learning-based}
A pretraining dataset has been mined from KEGG and the model has been constructed and is ready for generative training. % readiness
A dry-run of generative pre-training will be complete by Feb 16th. % dry run
Final model enhancements will be added prior to the pre-training wet-run, scheduled for Feb 16th. % optional improvements
% pretraining schedule
\par
A proof of concept lab screening experiment is ready to be done. I'm expecting this will take 1 week for all five mutants. Data will be processed into a training data set. % screening schedule
Training the model on the proof-of-concept screening dataset can start imediately after it becomes available.  % training schedule
The proof-of-concept pilot can be used to inform compound library and pool of mutants for screening. Alternatively a set of mutants and compounds can be designed right away. % proof of concept results
\par
A program for library selection can be ordered based on the results of the pilot or if purchasing restrictions permit it can be ordered right % library design readiness	

\subsubsection{Writing}
I am currently writing chapters for both approaches as papers. I will need some advice on overall features of the papers, depth of literature reveiws and data to include and data to exclude which will guide my experimental work. % progress - motivate papers
I will send you the draft PDFs for comment by Feb 16th. % comments
\par
I will expand the draft papers into thesis chapters from here up until submission. % expansion
By the end of the month I will have sent template papers for each project for comments and have a clear direction of where to expand into a thesis. % target
\subsubsection{Targets}
\begin{itemize}
\item Feb 16th - pre-train model dry run
\item Feb 16th - run enz evolution run
\item Feb 17th - design primers for enz mutants
\item Feb 16th - send draft papers for comment
\item Feb 16th - Start pilot screen
\item Feb 16th - Finish pilot screen
\item Feb 21st - implement flexible docking and loop remodelling in \textit{enz}
\item Feb 28th - Generate mutants for both projects - order primers
\item Feb 28th - Order compound library
\item Feb 28th - Get writing advice
\item Feb 28th - Set up LCMS pipeline
\item Feb 28th - Finish template papers for thesis writing
\end{itemize}
% papers - modifications and planning

\subsection{March}
\subsubsection{Structure-based}
In March I will prepare DNA stocks for mutants predicted using \textit{enz}. The maximum number of rounds of site-directed mutagenesis will be three, so this work can be fit into one month.  % lab - dna work
Any available slots for incubator and centrifuge time will be booked for next month. % prepare testing
I will create benchmarks for mesotrione product detection by LCMS using BM3 A82F/F87V I have on hand. This establishes a pipeline for upcoming product detection experiments. % lcms practice

\subsubsection{Machine learning-based}
At this point, a compound library must have been ordered. Allowing a lead time of one month, the screen will be ready for April. % order library
Primers should arrive early in the month. DNA work here will be  batched with DNA work for the \textit{enz} project. % design mutants
% lab - dna work
% expresion if possible
\subsubsection{Writing}
\textit{Computational methods:} % comp methods
I will write a detailed description for the \textit{rio} model, including relevant background.  % formalize model
I will write a detailed description of the \textit{enz} / genetic algorithm combination. % formalize enz
Methods can be written with input from Syngenta. % syngenta input

\subsubsection{Targets}
\begin{itemize}
\item March 1st - order compound library
\item March 1st - order primers for both sets of mutant 
\item March 30th - finish DNA work
\item March 30th - Benchmark LCMS experiment of mesotrione and BM3 A82F/F87V 
\item March 30th - Write computational methods for \textit{enz} and \textit{rio}
\end{itemize}

\subsection{April}
April will be dedicated to expression and purification of mutants from both projects. By batching expressions and purifications I can work on several  mutants concurrently. \textit{rio} mutants have higher priority. 
\subsection{Writing}
Write all lab methods.
\subsubsection{Targets}
\begin{itemize}
\item April 30th - Finish mutant expression 
\item April 30th - Have purified most \textit{rio} mutants
\item April 30th - have started all purifications
\item April 30th - finish writing all lab methods
\end{itemize}

\subsection{May}
\subsubsection{Structure-based}
\textit{enz} mutants finish purification in May. By the end of the month all \textit{enz} mutants should be purified and ready for testing. % purify mutants
Testing of these mutants takes place in batches of 3-4, to be completed by end of June. % test mutants

\subsubsection{Machine learning-based}
All \textit{rio} mutants are to be purified in the first weeks of May with higher priority than the \textit{enz} mutants.
The screen will happen in May. The screen can be done in 1 week, but I'm allowing 1 month for up to 8 mutants and 96 compounds for leeway. It'll be done in one batch as far as posible. % screen
Model training can begin as soon as the data is available. Some parameter searches will take place here, lasting one week.  % train 
The trained model can be deployed with an existing genetic algorithm to generate fit mutants immediately with indicated model certainty for each. Mutant generation will be constrained to within 3 mutations of existing DNA templates. Primers for site-directed mutagenesis to be ordered immediately. % generate mutants
\subsubsection{Writing}
Make any necessary retrospective changes to methods. Start preparing results section based on screen.
\subsubsection{Targets}
\begin{itemize}
\item May 30th - Finish screening
\item May 30th - Train model on screening data
\item May 30th - Predict new \textit{rio} mutants and order primers
\item May 30th - Finish purification of all \textit{enz} mutants, start testing
\end{itemize}

\subsection{June}
\subsubsection{Structure-based}
Test $K_d$, $K_{cat}$ and indication of metabolite production for all \textit{enz} mutants by end of month. Finish \textit{enz} lab work.
\subsubsection{Machine learning-based}
Express all \textit{rio} mutants by end of month. Begin purifications for all \textit{rio} mutants. Begin testing $K_d$, $K_{cat}$ and indication of metabolite production for all mutants by end of month.
\subsubsection{Writing}
Write results for \textit{enz} mutants. Write results so far for \textit{rio}.
\subsubsection{Targets}
\begin{itemize}
\item End of month - Finish all lab testing for \textit{enz}
\item End of month - Have begun all purifications for \textit{rio} mutants
\item End of month - Finish writing results for \textit{enz}
\end{itemize}


\subsection{July}
\subsubsection{Machine learning-based}
Purification and testing of \textit{rio} mutants must proceed as fast as possible in July, after which all lab work will finish.
\subsubsection{Writing}
I will write a discussion for \textit{enz} and  results and discussion for \textit{rio}. % enz and rio results and discussion
I will also start writing introductions for each peice of work. I aim to reach a minumum viable product (MVP) for each introduction by the end of the month.
\subsubsection{Targets}
\begin{itemize}
\item Finish lab testing of \textit{rio} mutants
\item End of month - End all lab work
\item End of month - Finish \textit{enz} results and discussion to MVP
\item End of month - Finish \textit{rio} results
\item End of month - MVP for both introductions
\end{itemize}

\subsection{August}
\subsubsection{Writing}
At this point, I should have finished writing methods and results sections for both projects and the discussion for \textit{enz}. % thesis so far
I will write a discussion for \textit{rio}. % rio discussion
I will finish introductions for each section and iteratively apply suggestions from readers. % comments
By the end of the month I will need a peice of work ready to submit. % ready to submit
\subsubsection{Targets}
\begin{itemize}
\item Finish \textit{rio} discussion
\item Finish \textit{rio} introduction
\item Finish \textit{enz} introduction
\item Ready to submit
\end{itemize}
\bibliography{plan}
\bibliographystyle{plain}
\end{document}
